% 
% SEGUNDO FORO INTERNACIONAL DE TALENTO MEXICANO
% INNOVATION MATCH MX 2016-2017
% Que se llevará a cabo los días: 31 Mayo, 1 y 2 de Junio 2017
% Fecha de Inicio de registro: 21 de Junio de 2016
% Fecha límite para recepción de resúmenes y registro de ponentes: 30 de Septiembre de 2016
% 
% INNOVATION MATCH MX es un punto de encuentro innovador donde empresas, investigadores y
% estudiantes pueden compartir conocimientos, experiencias y crear oportunidades de negocio con
% base en las demandas y ofertas tecnológicas de cada uno de los participantes.
% 
% Este foro está dirigido a una amplia temática en innovación, siendo los sectores estratégicos:
% 
% Ingeniería y manufactura (Electrónica, Robótica, Mecatrónica, Automotriz,
% Aeroespacial y Manufactura)
% 
% 
% En el registro se solicitarán los datos de la persona de contacto, así como historial académico,
% área de conocimiento, así como título y resumen del trabajo a postular
% 
% *****************
% El resumen deberá contener los siguientes elementos: 
% Propósito, 
% Metodología, 
% Resultados,
% Conclusión y 
% Aplicación Industrial o bien Oportunidad de Negocio.
% *****************
% 
% 
% 
% Con el fin de impulsar la participación de los investigadores y estudiantes mexicanos de todo el
% mundo, INNOVATION MATCH MX otorgará becas para viáticos y viajes a aquellos participantes que
% lo soliciten y justifiquen la necesidad del apoyo.
% 
% El resumen deberá tener una extensión no mayor a las 250 palabras a espacio sencillo, Arial 12,
% redactado en inglés y español. Es requisito indispensable presentar el título y resumen en ambos
% idiomas.
% 
% 

% \documentclass[10pt,journal,compsoc]{IEEEtran}
\documentclass[12pt,journal,onecolumn,compsoc]{IEEEtran}


 
 


\usepackage[switch]{lineno}
% \linenumbers
% \leftlinenumbers
% % % http://tex.stackexchange.com/questions/20368/lineno-for-2-column






% *** CITATION PACKAGES ***
%
\ifCLASSOPTIONcompsoc
  % IEEE Computer Society needs nocompress option
  % requires cite.sty v4.0 or later (November 2003)
  \usepackage[nocompress]{cite}
\else
  % normal IEEE
  \usepackage{cite}
\fi




% *** GRAPHICS RELATED PACKAGES ***
%
\ifCLASSINFOpdf
  \usepackage[pdftex]{graphicx}
   \graphicspath{{figures/}}

\else

\fi






% *** MATH PACKAGES ***
%
\usepackage[cmex10]{amsmath}
\usepackage{amsfonts} % to use $\mathbb{Z}$


% *** SPECIALIZED LIST PACKAGES ***

\usepackage{algorithm}
\usepackage{algorithmic}





% *** PDF, URL AND HYPERLINK PACKAGES ***
%
\usepackage{url}




% correct bad hyphenation here
\hyphenation{op-tical net-works semi-conduc-tor  Birmingham}


\begin{document}


\title{
Towards the improvement of Healthy Ageing with Humanoid Robots
}


% \title{
% Improving Healthy Ageing with \\ Humanoid Robots
% }

% \title{
% Using NAO Robot to improve \\  the quality of life in the elderly
% }



% “IMPROVING THE QUALITY OF LIFE IN THE ELDERLY USING ROBOTIC 
% ASSISTIVE TECHNOLOGY: BENEFITS, LIMITATIONS, AND CHALLENGES”



\author{
 	Miguel~Xochicale,~\IEEEmembership{Doctoral~Researcher}
%          Chris~Baber,~\IEEEmembership{Lead~Supervisor;}
%          ~Martin~Russell,~\IEEEmembership{Co-Supervisor;} \\
%          and~?~?,~\IEEEmembership{Academic~Advisor.}
         % <-this % stops a space

 \IEEEcompsocitemizethanks{\IEEEcompsocthanksitem 
 M. Xochicale is with the School of Electronic, Electrical and Systems Engineering, 
 The University of Birmingham, U.K. \protect\\
 % note need leading \protect in front of \\ to get a newline within \thanks as
 % \\ is fragile and will error, could use \hfil\break instead.
 E-mail: see http://mxochicale.github.io/
 }% <-this 
% stops an unwanted space
% \thanks{Manuscript received August 15, 2016; revised Month Day, 2016.}
}



% The paper headers
\markboth{Abstract Submission to Innovation Match Mx 2016-2017}%
{Shell \MakeLowercase{\textit{et al.}}: Bare Demo of IEEEtran.cls for Computer 
Society Journals}
% The only time the second header will appear is for the odd numbered pages
% after the title page when using the twoside option.
% 


\IEEEtitleabstractindextext{%


 
% \begin{abstract}
% ...
% \end{abstract}


% Note that keywords are not normally used for peerreview papers.
\begin{IEEEkeywords}
Engineering; Robotics; Health Sciences
\end{IEEEkeywords}


}



% make the title area
\maketitle


\IEEEdisplaynontitleabstractindextext

\IEEEpeerreviewmaketitle





\IEEEraisesectionheading{\section*{Abstract}\label{sec:introduction}}


In 2015, it was estimated that 125 million people worldwide were aged 80 years or older.
By 2050, it is predicted that 350 million of older people will live in low- and middle-income countries \cite{AH16}. 
%  http://www.who.int/mediacentre/factsheets/fs404/en/
It is worthwhile to mention that some of the general key environmental factors to have a healthy ageing are
long-term care, good relationships with friends, family and care givers.
% Global strategy and action plan on ageing and health (2016- 2020)
% http://www.who.int/ageing/GSAP-Summary-EN.pdf?ua=1
To accomplish the previous factors, 
the World Health Organization mentions 
that one of the main challenges to create a Healthy Ageing
is the  improvement of methodologies for measurement, monitoring and understanding  
many ageing problems.
% However, there are still many challenges on Healthy Ageing
% in order to have 
% better understanding of the elderly in real life situations
% in order to create the appropriate methods for intervertion and help of the elder.


% after high blood pressure (12.8\%) , tobacco use (8.7\%), high blood glucose (5.8\%) and 
% right before overweight and obesity (4.8\%) \cite{GHR09}.
% Physical inactivity is estimated to cause around 
% 21–25\% of breast and colon cancer burden, 27% of 
% diabetes and about 30\% of ischaemic heart disease 
% burden.
%  World Health Organization. Global health risks: mortality and burden of diseases attributable to selected major
%  risks. WHO, Geneva (2009)
% % http://www.who.int/healthinfo/global_burden_disease/global_health_ris ks/en/index.html.
% % http://www.who.int/healthinfo/global_burden_disease/GlobalHealthRisks_report_full.pdf
% % ttp://www.who.int/healthinfo/global_burden_disease/global_health_risks/en/
% % 

% The predictions for 2050 of elderly people will hist almost 
% one million people in Japan, similarly the projection for elderly people
% in 2050 will be around  19 million.
I am therefore proposing the use of Humanoid Robots to have a Healthy Ageing and to encourage the elderly
to have the proper amount of physical activity. 
Elderly care using Robots has been well developed, mainly, in Japan. For instance,
(a) Ri-Man can see, hear and assess a person's health; (b) Paro therapy bot help people with dementia; 
(c) Palro humanoid robot can play games and dance, to mention but a few.
Similarly, humanoid robots such as Pepper and NAO have been used to understand the emotions of people,
to play football or to play games with humans. In the case of health applications, 
NAO has been used to teach diabetic children about various aspects their condition.
Also, NAO has been used for arm rehabilitation therapy for children for which children found the Robot 
interaction activity as one which is more engaging 
and increase the motivation of children to perform an adequate rehabilitation therapy.


For this work, I believe that it is required to create sufficient physical activity for the elderly
given that insufficient physical activity is the fourth leading global risk for mortality in the world with 5.5\%  
\cite{GHR09}. 

Therefore, I am proposing the use of Robots to improve the quality of life of elderly persons
with regard to the encouragement of performing physical activity.
I am planning to present preliminary outcomes of the use of NAO as a instructor
for participants using on-body worn sensors 
to copy movements in scenarios for entertainment and rehabilitation.
I will present the advances and disadvantages of using on-body inertial sensors,
methodologies for data processing and the measure of the quality of activities 
within and across participants.

% I am going to present a literature review of robots for elderly care, areas of care, etc.

Finally, I will pointed out to the Mexican community that 
Humanoids Robots can provide recreation services to the elderly by playing games and dancing with them
in order to measure, analise, understand and to improve the health of the elderly.


% http://tribune.com.pk/story/1138709/charlie-robot-new-best-buddy-kids-diabetes/
% Towards Long-Term Social Child-Robot Interaction:
% Using  Multi-Activity  Switching  to  Engage  Young
% Users
% Alexandre Coninx



% * JAPAN
% t's expected to hit almost 1 million by 2050. 
% https://www.cnet.com/news/fitness-bot-whips-japanese-seniors-into-shape/

% * UK
% 10 million people in the UK are over 65 years old.  
% The latest projections are for 5½ million more elderly people in 
% 20 years time and the number will have nearly doubled to around 19 million by 2050.
% http://www.parliament.uk/business/publications/research/key-issues-for-the-new-parliament/value-for-money-in-public-services/the-ageing-population/





% \cite{Lorenzi2016}







% 
% % use section* for acknowledgment
% \ifCLASSOPTIONcompsoc
%   % The Computer Society usually uses the plural form
%   \section*{Acknowledgments}
% \else
%   % regular IEEE prefers the singular form
%   \section*{Acknowledgment}
% \fi
% 
% Miguel Xochicale gratefully acknowledges the studentship from 
% the National Council for Science and Technology (CONACyT) Mexico
% to pursue his postgraduate studies at University of Birmingham U.K.
% 
% \ifCLASSOPTIONcaptionsoff
%   \newpage
% \fi



% trigger a \newpage just before the given reference
% number - used to balance the columns on the last page
% adjust value as needed - may need to be readjusted if
% the document is modified later
%\IEEEtriggeratref{8}
% The "triggered" command can be changed if desired:
%\IEEEtriggercmd{\enlargethispage{-5in}}

% references section

% can use a bibliography generated by BibTeX as a .bbl file
% BibTeX documentation can be easily obtained at:
% http://www.ctan.org/tex-archive/biblio/bibtex/contrib/doc/
% The IEEEtran BibTeX style support page is at:
% http://www.michaelshell.org/tex/ieeetran/bibtex/
%\bibliographystyle{IEEEtran}
% argument is your BibTeX string definitions and bibliography database(s)
%\bibliography{IEEEabrv,../bib/paper}
%
% <OR> manually copy in the resultant .bbl file
% set second argument of \begin to the number of references
% (used to reserve space for the reference number labels box)
% \begin{thebibliography}{1}
% 
% \bibitem{IEEEhowto:kopka}
% H.~Kopka and P.~W. Daly, \emph{A Guide to \LaTeX}, 3rd~ed.\hskip 1em plus
%   0.5em minus 0.4em\relax Harlow, England: Addison-Wesley, 1999.
% 
% \end{thebibliography}

% \nocite{*}
\bibliographystyle{IEEEtran}
\bibliography{references}


% biography section
% 
% If you have an EPS/PDF photo (graphicx package needed) extra braces are
% needed around the contents of the optional argument to biography to prevent
% the LaTeX parser from getting confused when it sees the complicated
% \includegraphics command within an optional argument. (You could create
% your own custom macro containing the \includegraphics command to make things
% simpler here.)
%\begin{IEEEbiography}[{\includegraphics[width=1in,height=1.25in,clip,keepaspectratio]{mshell}}]{Michael Shell}
% or if you just want to reserve a space for a photo:

% \begin{IEEEbiography}[{\includegraphics[width=1in,height=1.25in,clip,keepaspectratio]{mxochicale38x44.pdf}}]{name}

% \begin{IEEEbiography}{Miguel Perez-Xochicale}
% ........................
% \end{IEEEbiography}



% % if you will not have a photo at all:
% \begin{IEEEbiographynophoto}{John Doe}
% Biography text here.
% \end{IEEEbiographynophoto}
% 
% % insert where needed to balance the two columns on the last page with
% % biographies
% %\newpage
% 
% \begin{IEEEbiographynophoto}{Jane Doe}
% Biography text here.
% \end{IEEEbiographynophoto}

% You can push biographies down or up by placing
% a \vfill before or after them. The appropriate
% use of \vfill depends on what kind of text is
% on the last page and whether or not the columns
% are being equalized.

%\vfill

% Can be used to pull up biographies so that the bottom of the last one
% is flush with the other column.
%\enlargethispage{-5in}



% that's all folks
\end{document}
